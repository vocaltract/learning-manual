\documentclass[UTF8]{ctexart}
%\documentclass[a4paper]{article}
% \usepackage[margin=1.25in]{geometry}
\usepackage[inner=2.0cm,outer=2.0cm,top=2.5cm,bottom=2.5cm]{geometry}
%\usepackage{CJK}
\usepackage{color}
\usepackage{graphicx}
\usepackage{amssymb}
\usepackage{amsmath}
\usepackage{amsthm}
\usepackage{bm}
\usepackage{hyperref}
\usepackage{multirow}
\usepackage{enumerate}
\usepackage{listings}
\usepackage{xcolor}
\usepackage{fontspec}
\setmainfont{Times New Roman}

\newcommand{\homework}[5]{
    \pagestyle{myheadings}
    \thispagestyle{plain}
    \newpage
    \setcounter{page}{1}
    \noindent
    \begin{center}
    \framebox{
        \vbox{\vspace{2mm}
        \hbox to 6.28in { {\bf 智能系统设计与应用 \hfill #2} }
        \vspace{6mm}
        \hbox to 6.28in { {\Large \hfill \bf #1 \hfill} }
        \vspace{6mm}
        \hbox to 6.28in { {\it 指导老师: {\rm #3} \hfill 姓名: {\rm #4},学号: {\rm #5}}}
        \vspace{2mm}}
    }
    \end{center}
    % \markboth{#4 -- #1}{#4 -- #1}
    \vspace*{4mm}
}


\newcommand{\yahei}{\setCJKfamilyfont{yahei}{Microsoft YaHei} \CJKfamily{yahei}}




\newenvironment{solution}
{\color{blue} \paragraph{Solution.}}
{\newline \qed}

\begin{document}


%大标题在这里改!!!!!!!!!
\homework{课后练习}{2020 春}{章宗长}{王晨渊}{181220057}

\lstset{numbers=left,numberstyle=\tiny,keywordstyle=\color{blue!70},commentstyle=\color{red!50!green!50!blue!50},frame=shadowbox, rulesepcolor=\color{red!20!green!20!blue!20},escapeinside=``,xleftmargin=2em,xrightmargin=2em, aboveskip=1em}

\section{}
\subsection{用你自己的话定义下列术语:
Agent,Agent函数,Agent程
序,理性,自主,反射Agent,基于模型的Agent,基于目标
的Agent,基于效用的Agent,学习Agent。}

\subsection{请写出基于目标的Agent和基于效用的Agent的伪代码Agent
程序。}

\section{}
\subsection{在火星上,有50\%的概率既有生命又有水,有25\%的概率有
生命但没有水,有25\%的概率既没有生命又没有水。问:在
给定有水的前提下,火星上有生命的概率是多少?}
\begin{solution}
令事件$A:$火星上有水 $B:$火星上有生命

已知$P(A,B)=0.5,P(\overline{A},B)=0.25,P(\overline{A},\overline{B})=0.25$,求$P(B|A)$

$P(A)=1-P(\overline{A})=1-(P(\overline{A},\overline{B})+P(\overline{A},B))=0.5$

$P(B|A)=\frac{P(A,B)}{P(A)}=1$

所以在给定有水的前提下,火星上有生命的概率为100\%
\end{solution}
\subsection{给定一个状态序列为$S_{0:t}$,和观察序列为$O_{0:t}$的隐马尔科夫模型,试证明:}

\begin{equation}
    P(S_t|O_{0:t})\propto P(O_t|S_t,O_{0:t-1})P(S_t|O_{0:t-1})
\end{equation}

\textbf{请用上式证明:}
\begin{equation}
    P(S_t|O_{0:t})\propto P(O_t|S_t)\sum_{S_{t-1}}P(S_t|S_{t-1})P(S_{t-1}|O_{0:t-1})
\end{equation}

\begin{solution}
    
    $\because P(S_t|O_{0:t})=\frac{P(O_t,S_t|O_{0:t-1})}{P(O_t)} \propto P(O_t,S_t|O_{0:t-1})=P(O_t|S_t,O_{0:t-1})P(S_t|O_{0:t-1})$

    $\therefore P(S_t|O_{0:t})\propto P(O_t|S_t,O_{0:t-1})P(S_t|O_{0:t-1})$

    $\because (O_t\perp O_{0:t-1}|S_t),(S_t\perp O_{0:t-1}|S_{t-1})$

    $\therefore P(O_t|S_t)=P(O_t|S_t,O_{0:t-1})$
    $,P(S_t|S_{t-1})=P(S_t|S_{t-1},O_{0:t-1})$

    $\therefore P(O_t|S_t,O_{0:t-1})P(S_t|O_{0:t-1})$
    
    $=P(O_t|S_t)P(S_t|O_{0:t-1})$

    $=P(O_t|S_t)\sum\limits_{S_{t-1}}P(S_t,S_{t-1}|O_{0:t-1})$

    $=P(O_t|S_t)\sum\limits_{S_{t-1}}P(S_t|S_{t-1},O_{0:t-1})P(S_{t-1}|O_{0:t-1})$

    $=P(O_t|S_t)\sum\limits_{S_{t-1}}P(S_t|S_{t-1})P(S_{t-1}|O_{0:t-1})$

    $\therefore P(S_t|O_{0:t})\propto P(O_t|S_t)\sum_{S_{t-1}}P(S_t|S_{t-1})P(S_{t-1}|O_{0:t-1})$
\end{solution}






\subsection{什么是分类任务?朴素贝叶斯模型使用了什么假设?使用盘式记法来画一个朴素的贝叶斯模型。}
\begin{solution}
    分类任务是指从给定的一组观察或特征中推断样本所属的类别。

    朴素贝叶斯的假设是:给定所属的类别,所有证据变量之间彼此条件独立。

    模型如图(用PPT画的)
\end{solution}

\begin{figure}[ht]
    \centering
    \includegraphics[scale=0.4]{bayes.jpg}%还有
    \caption{简单的朴素贝叶斯模型}
\end{figure}

\subsection{有一位教授想知道学生是否睡眠充足。每天,教授观察学生
在课堂上是否睡觉,并观察他们是否红眼。教授有如下的领
域理论:}
\begin{enumerate}[1)]
    \item 没有观察数据时,学生睡眠充足的先验概率为0.7。
    \item 给定学生前一天睡眠充足为条件,学生在晚上睡眠充足的概率是0.8;
    如果前一天睡眠不充足,则是0.3。
    \item 如果学生睡眠充足,则红眼的概率是0.2,否则是0.7。
    \item 如果学生睡眠充足,则在课堂上睡觉的概率是0.1,否则是0.3。
\end{enumerate}
\phantom{空格}\textbf{将这些信息形式化为一个动态贝叶斯网络,使教授可以使用
这个网络从观察序列中进行滤波和预测。然后再将其形式化
为一个只有一个观察变量的隐马尔科夫模型。给出这个模型
的完整的概率表。}

\subsection{对于上一道练习描述的动态贝叶斯网络以及证据变量值}
\begin{enumerate}[1)]
    \item $O_0:$没有红眼,没有在课堂上睡觉
    \item $O_1:$有红眼,没有在课堂上睡
    \item $O_2:$有红眼,在课堂上睡
\end{enumerate}
\textbf{执行下面的计算:}
\begin{enumerate}[a)]
    \item 状态估计:针对每个$t = 0, 1, 2$,计算$P(EnoughSleep_t|O_{0:t})$
    \item 平滑:针对每个$t = 0, 1, 2$,计算$P(EnoughSleep_t|O_{0:2})$
    \item 针对$t=0$和$t=1$,比较滤波概率和平滑概率
\end{enumerate}
\begin{solution}
    根据我对于本题目的理解,我认为今天是否红眼或课上睡觉仅与前一天晚上的睡眠有关。


    令$S_t=\mathbf{I}\{$学生在第$t$天晚上睡眠充足$\},R_t=\mathbf{I}\{$学生在第$t+1$天红眼$\},A_t=\mathbf{I}\{$学生在第$t+1$天的课上睡觉$\}$
    ,其中$\mathbf{I}$为指示函数。

    值得注意的是,我的定义中特地调整了一天使其更吻合我们通常的模式。

    由于定义的问题,我觉得自己的答案可能会与其他人不太一样,求老师能仔细看下。 

    已知$P(S_1=1)=0.7,P(S_t=1|S_{t-1}=1)=0.8,P(S_t=1|S_{t-1}=0)=0.3$

    $P(R_t=1|S_{t}=1)=0.2,P(R_t=1|S_{t}=0)=0.7$

    $P(A_t=1|S_{t}=1)=0.1,P(A_t=1|S_{t}=0)=0.3$

    其对应的动态贝叶斯网络图如下(PPT画的)

    \begin{figure}[ht]
        \centering
        \includegraphics[scale=0.2]{init.png}%还有
        \caption{简单的朴素贝叶斯模型}
    \end{figure}



    令随机变量$O_t$有如下性质

    \[ 
    O_t= 
    \begin{cases}
        0 &  R_t=0 \wedge A_t=0\\ 
        
        1 & R_t=1 \wedge A_t=1\\
        
        2 &  R_t=0 \wedge A_t=1\\

        3 & R_t=1 \wedge A_t=0\\
    \end{cases} 
    \] 

    从而我们将其转化为隐马尔科夫模型如下



    其完整概率表为


Q
\end{solution}


\end{document}
